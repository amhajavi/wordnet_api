\documentclass{beamer}
% Replace the \documentclass declaration above
% with the following two lines to typeset your
% lecture notes as a handout:
%\documentclass{article}
%\usepackage{beamerarticle}
\usepackage{mathtools,amssymb, amsfonts}
\usetheme{Marburg}
\setbeamersize{sidebar width right=2cm}
\setbeamertemplate{sidebar right}
{
  {\usebeamerfont{title in sidebar}%
    \vskip1.5em%
    \usebeamercolor[fg]{title in sidebar}%
    \insertshorttitle[width=1.9cm,center,respectlinebreaks]\par%
    \vskip1.25em%
  }%
  \hbox to2cm{\hss\insertlogo\hss}
  \vskip1.25em%
  \insertverticalnavigation{\2cm}%
  \vfill
  \hbox to2cm{\hskip0.8cm\usebeamerfont{subsection in
      sidebar}\strut\usebeamercolor[fg]{subsection in
      sidebar}\insertframenumber/\inserttotalframenumber\hfill}%
  \vskip3pt%
}%



\title{Some Basics over Fars-net and Word-net}
\subtitle{}
\author{Amir Hossein Hajavi}
\institute[University of Tehran]{ % (optional, but mostly needed)
University of Tehran
}
\date{Course of Natural Language Processing, winter 2015}
\subject{Combinatorics On words}

 \AtBeginSubsection[]
{
	\begin{frame}<beamer>{Outline}
	\tableofcontents[currentsection,currentsubsection]
	\end{frame}
}

 \begin{document}

 \begin{frame}
\titlepage
\end{frame}

 \begin{frame}{Outline}
\tableofcontents
% You might wish to add the option [pausesections]
\end{frame}

 % Section and subsections will appear in the presentation overview
% and table of contents.
\section{Farsnet}

\subsection{What Farsnet Does}

\begin{frame}{Farsnet}{What Farsnet Does}
	\begin{itemize}
		\item<1-> {	Farsnet is the most known Persian Word-net. }
   	    \item<2-> {	It covers all type of words with relations between them.}
		\item<3-> {	It contains over 15000 words in 10000 synsets (explained later on)}
		\item<4-> {	It has been created semi-automatically.}
	\end{itemize}
\end{frame}

\subsection{How is it related to other Word-nets}	

\begin{frame}{Farsnet}{How is it related to other Word-nets}
	\begin{itemize}
		\item<1-> { Relations are same as Princton's Word-net. }
		\item<2-> {	Farsnet's core is a translation of Balkan Word-net. }
		\item<3-> {	It makes some relations over synsets to other Word-net for translation purposes. }
	\end{itemize}
\end{frame}

\section{Word-nets}

\subsection{How they're made}

\begin{frame}{Word-nets}{How they're made}
	\begin{itemize}
		\item<1-> {
			There are 4 type of words which play important roles.
		}
		\item<2-> {
			In dictionaries all of them are ordered alphabetically.
		}
		\item<3-> {
			Word-nets divide them in synsets (each synset has only one syntactic role. )
		}
	\end{itemize}
\end{frame}

\begin{frame}{Word-nets}{How they're made}

	\begin{itemize}
		\item<1-> {
				All word forms represent a word meaning.
			}
		\item<2-> {
				Some words have several meanings. 
			}
		\item<3->{
				Some meaning can be presented with several words. 	
			}
	\end{itemize}
\end{frame}

\begin{frame}{Word-nets}{How they're made}
	Psycholinguistics use the box model to illustrate the relationship between word and meaning.  
	\begin{itemize}
		\item<1-> {
				A box for meanings.
			}
		\item<2-> {
				A box for word forms.
			}
		\item<3->{
				Arrows between these two boxes to relate the meanings.
			}
	\end{itemize}
\end{frame}

\begin{frame}{Word-nets}{How they're made}
	We use a different kind of model but still the same.
	\begin{itemize}
		\item<1-> {
				All word forms are as column heads of a matrix.
			}
		\item<2-> {
				All word meanings are as row initials of the matrix
			}
		\item<3->{
				A cell of matrix is 1 if the corresponding word presents the corresponding meaning otherwise it's 0
			}
	\end{itemize}
\end{frame}

\begin{frame}{Word-nets}{How they're made}
	Now to make it a network.
	\begin{itemize}
		\item<1-> {
				Make each word form a node.
			}
		\item<2-> {
				Consider a row in this matrix and draw out cells that are 1 in it.
			}
		\item<3->{
				Add a relation of synonymy between every node which is in those drawn out words. 	
			}
		\item<3->{
				There would be disconnected subgraphs of words afterwards.
			}
	\end{itemize}
\end{frame}

\begin{frame}{Word-nets}{How they're made}
	Now to make it a network.
	\begin{itemize}
		\item<1-> {
				Each pair of meanings have their own relation.
			}
		\item<2-> {
				Every pair of word forms follow their meanings in relations
			}
		\item<3->{
				Adding those relations together makes the subgraphs as a whole.	
			}
	\end{itemize}
\end{frame}

\subsection{How they're represented}

\begin{frame}{Word-nets}{How they're represented}
	Now to make it a network.
	\begin{itemize}
		\item<1-> {
				Showing these graphs can be different.
			}
		\item<2-> {
				Since we only need one relation over this graph it can be a tree form.
			}
		\item<3->{
				Hence best tool for trees comes as XML files. Word-nets usually go in XML forms
			}
	
	\end{itemize}
\end{frame}

\section{Farsnet Semi Automation}

\subsection{Automation}
\begin{frame}{Farsnet}{Automation}
	Automation of Farsnet with various assists.
	\begin{itemize}
		\item<1-> {
				Using morphological rules to expand relations.
			}
		\item<2-> {
				Using syntactical occurrences for adjectives (From PLDB and Bijankhan). 
			}
		\item<3->{
				Using wikipedia pages and analyzing them. 
			}
	\end{itemize}
\end{frame}

\subsection{Making it semi}
\begin{frame}{Farsnet}{Semi Automation}
	Using Dictionaries
	\begin{itemize}
		\item<1-> {
				checking the received results with monolingual dictionaries like "SOKHAN".
			}
	\end{itemize}
\end{frame}

\end{document}